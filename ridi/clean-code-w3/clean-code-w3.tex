% !TEX encoding = UTF-8 Unicode

\documentclass{beamer}

\usepackage{kotex}
\usepackage{color}
\usepackage{listings}
\lstset{
    basicstyle=\tiny
}

\title{Clean Code}
\subtitle{3주차: 형식 맞추기 \& 객체와 자료구조}
\author{서지혁}
\institute{리디북스 스토어팀}
\date{2017년 5월 17일}

\begin{document}

\frame{\titlepage}

\begin{frame}
    \frametitle{형식 맞추기의 목적}
    \framesubtitle{형식 맞추기, 왜 하는가?}
    \begin{itemize}
        \item because \emph{important!}
        \item 형식 맞추기는 의사 소통을 위한 것
        \item 코딩 스타일과 가독성이 곧 지속 가능성과 확장 가능성
        \item 그래서, 형식을 어떻게 맞춰야 의사 소통을 잘 할 수 있을까?
    \end{itemize}
\end{frame}

\begin{frame}
    \frametitle{형식 맞추기 (Formatting)}
    \begin{itemize}
        \item 수직 포매팅
        \item 수평 포매팅
        \item `밥' 형님의 포매팅 규칙
        \item 팀 규칙
    \end{itemize}
\end{frame}

\begin{frame}
    \frametitle{수직 포매팅 (Vertical Formatting)}
    \begin{itemize}
        \item `신문' 비유
        \item 수직 공간
        \item 수직 밀도
        \item 수직 거리
        \item 수직 순서
    \end{itemize}
\end{frame}

\begin{frame}
    \frametitle{`신문' 비유} 
    \begin{itemize}
        \item 코드를 잘 짜여진 신문 기사처럼 짜자
        \item 파일명만으로도 우리가 보고 싶은 파일인지 알 수 있게
        \item 파일 시작 부분에는 고수준의 개념과 알고리즘
        \item 아래로 내려갈수록 디테일하게
    \end{itemize}
\end{frame}

\begin{frame}
    \frametitle{수직 공간}
    \begin{itemize}
        \item 거의 모든 코드는 왼쪽에서 오른쪽으로, 위에서 아래로 읽힌다
        \item 코드 한 줄 한 줄은 하나의 완전한 사고를 이루어야 함
        \item 각각의 이러한 사고들은 빈 줄로 분리되어야 함
    \end{itemize}
\end{frame}

\begin{frame}
    \frametitle{수직 공간}
    \framesubtitle{Clean Code}
    \lstinputlisting[language=Java]{clean/BoldWidget.java}
\end{frame}

\begin{frame}
    \frametitle{수직 공간}
    \framesubtitle{Dirty Code}
    \lstinputlisting[language=Java]{dirty/BoldWidget.java}
\end{frame}

\begin{frame}
    \frametitle{수직 밀도}
    \begin{itemize}
        \item 수직 공간이 개념을 분리한다면
        \item 수직 밀도는 밀접한 관계를 다룬다
        \item 쓸 데 없는 내용으로 밀접해야 할 관계를 벌리지 말자
    \end{itemize}
\end{frame}

\begin{frame}
    \frametitle{수직 밀도}
    \framesubtitle{Clean Code}
    \lstinputlisting[language=Java]{clean/ReporterConfig.java}
\end{frame}

\begin{frame}
    \frametitle{수직 밀도}
    \framesubtitle{Dirty Code}
    \lstinputlisting[language=Java]{dirty/ReporterConfig.java}
\end{frame}

\begin{frame}
    \frametitle{수직 거리}
    \begin{itemize}
        \item 코드의 의도를 읽을 시간에 코드의 위치를 찾게 하지 말자
        \item 연관된 코드는 수직 거리를 최소한으로, 서로 가까이 두자
    \end{itemize}
\end{frame}

\begin{frame}
    \frametitle{수직 거리: 변수 선언}
    \framesubtitle{변수는 사용처와 최대한 가까이에 선언하기}
    \lstinputlisting[language=Java]{clean/readPreferences.java}
\end{frame}

\begin{frame}
    \frametitle{수직 거리: 변수 선언}
    \framesubtitle{루프 제어 변수는 루프 선언문 안에서 선언하기}
    \lstinputlisting[language=Java]{clean/countTestCases.java}
\end{frame}

\begin{frame}
    \frametitle{수직 거리: 인스턴스 변수}
    \framesubtitle{한 장소에 모아 두자}
    \lstinputlisting[language=Java]{clean/TestSuite.java}
\end{frame}

\begin{frame}
    \frametitle{수직 거리: 의존 함수}
    \framesubtitle{호출되는 함수는 호출하는 함수 아래 가까이}
    \lstinputlisting[language=Java]{clean/WikiPageResponder.java}
\end{frame}

\begin{frame}
    \frametitle{수직 거리: 개념적 유사성}
    \framesubtitle{유사한 개념일수록 수직 거리를 가까이}
    \lstinputlisting[language=Java]{clean/Assert.java}
\end{frame}

\begin{frame}
    \frametitle{수직 순서}
    \begin{itemize}
        \item 함수 호출 의존성이 아래로 향하게
        \item 호출되는 함수가 호출하는 함수 아래에
        \item 신문 기사처럼, 디테일을 아래쪽에 두자
    \end{itemize}
\end{frame}

\begin{frame}
    \frametitle{수평 포매팅 (Horizontal Formatting)}
    \begin{itemize}
        \item 수평 밀도
        \item 수평 정렬
        \item 들여쓰기
        \item 더미 스코프
    \end{itemize}
\end{frame}

\begin{frame}
    \frametitle{수평 밀도}
    \framesubtitle{수평 밀도를 통해 개념의 밀접도를 구분}
    \lstinputlisting[language=Java]{clean/measureLine.java}
    \lstinputlisting[language=Java]{clean/Quadratic.java}
\end{frame}

\begin{frame}
    \frametitle{수평 정렬}
    \framesubtitle{Dirty Code}
    \lstinputlisting[language=Java]{dirty/FitNesseExpediter.java}
\end{frame}

\begin{frame}
    \frametitle{수평 정렬}
    \framesubtitle{Clean Code}
    \lstinputlisting[language=Java]{clean/FitNesseExpediter.java}
\end{frame}

\begin{frame}
    \frametitle{들여쓰기}
    \framesubtitle{Dirty Code}
    \lstinputlisting[language=Java]{dirty/FitNesseServer.java}
\end{frame}

\begin{frame}
    \frametitle{들여쓰기}
    \framesubtitle{Clean Code}
    \lstinputlisting[language=Java]{clean/FitNesseServer.java}
\end{frame}

\begin{frame}
    \frametitle{더미 스코프}
    \framesubtitle{Clean Code}
    \lstinputlisting[language=Java]{clean/dummyscope.java}
    \begin{itemize}
        \item 가끔 while이나 for문의 내용이 더미일 때가 있음
        \item 웬만해선  이런 코드를 작성하는 건 피하자
        \item 꼭 필요할 땐 루프라는 사실을 명확히 하자
    \end{itemize}
\end{frame}

\begin{frame}
    \frametitle{`밥' 형님의 포매팅 규칙}
\end{frame}

\begin{frame}
    \frametitle{팀 규칙}
    \begin{itemize}
        \item 누구나 자기만의 취향이 있지만
        \item 좋은 프로그래머 = 팀 플레이어
        \item 팀 단위의 규칙을 IDE 코드 포매터에 적용
        \item 가장 중요한 것은 일관성을 지키는 것
    \end{itemize}
\end{frame}

\begin{frame}
    \frametitle{객체와 자료구조}
    \begin{itemize}
        \item 구현 디테일이 노출되고, 그것에 의존하는 코드가 생기는 것을 피하기 위해  프라이빗 변수를 사용한다
        \item 그럼 대체 왜 객체에 getter와 setter를 만들어서 프라이빗 변수를 노출시키나요?
        \item 다음 페이지에! 공개됩니다
    \end{itemize}
\end{frame}

\begin{frame}
    \frametitle{객체와 자료구조: 데이터 추상화}
    \framesubtitle{구현을 숨기는 건 추상화를 위한 것이다}
    \lstinputlisting[language=Java]{objects/ConcretePoint.java}
    \lstinputlisting[language=Java]{objects/AbstractPoint.java}
    \begin{itemize}
        \item 두 번째 예제가 아름다운 이유는 구현이 사각형인지 극좌표인지 모르지만, 분명 자료구조를 표현하기 때문
        \item 더 나아가, 자료구조 그 이상을 표현한다:
            \begin{itemize}
                \item 접근 권한 설정
                \item 각 좌표를 따로 읽을 수 있지만
                \item 좌표 설정은 한꺼번에 같이 해야 한다
            \end{itemize}
        \item 구현을 숨기는건 변수를 메서드로 숨기기 위한 것이 아니라 추상화를 위한 것
    \end{itemize}
\end{frame}

\begin{frame}
    \frametitle{객체와 자료구조: 데이터/객체의 비대칭성}
    \framesubtitle{Procedural: 새 도형을 추가하고 싶다면 Geometry 클래스 함수를 모두 고쳐야 한다}
    \lstinputlisting[language=Java]{objects/ProceduralShape.java}
\end{frame}

\begin{frame}
    \frametitle{객체와 자료구조: 데이터/객체의 비대칭성}
    \framesubtitle{Polymorphic: 새 함수를 추가하고 싶다면 도형 클래스 전부를 고쳐야 한다}
    \lstinputlisting[language=Java]{objects/PolymorphicShape.java}
\end{frame}

\begin{frame}
    \frametitle{객체와 자료구조: 데이터/객체의 비대칭성}
    \begin{itemize}
        \item 객체는 데이터를 추상화 뒤에 숨기고, 데이터를 조작하기 위한 함수를 노출한다
        \item 자료구조는 데이터를 노출하고, 별다른 기능이 없다
        \item 객체와 자료구조는 서로를 보완하는 관계
        \item 두 개의 차이와 각각의 장단점을 아는 것이 중요
    \end{itemize}
\end{frame}

\begin{frame}
    \frametitle{객체와 자료구조: 디미터의 법칙}
    \framesubtitle{모듈은 자신이 조작하는 객체의 내부 구현을 몰라야 한다}
    ``클래스 C의 메소드 f는 다음 객체의 메소드만 호출해야 한다''
    \begin{itemize}
        \item 클래스 C
        \item f가 생성한 객체
        \item f 인수로 넘어온 객체
        \item C 인스턴스 변수에 저장된 객체
    \end{itemize}
\end{frame}

\begin{frame}
    \frametitle{객체와 자료구조: 디미터의 법칙}
    \framesubtitle{모듈은 자신이 조작하는 객체의 내부 구현을 몰라야 한다}
    기차 충돌:
    \lstinputlisting[language=Java]{demeter/1.java}
    코드 나누기:
    \lstinputlisting[language=Java]{demeter/2.java}
    디미터 법칙을 거론할 필요가 없는 코드:
    \lstinputlisting[language=Java]{demeter/3.java}
    반 객체 반 자료구조인 잡종 구조를 다룰 때 머리가 아파지기 시작한다...
\end{frame}

\begin{frame}
    \frametitle{객체와 자료구조: 디미터의 법칙}
    \framesubtitle{모듈은 자신이 조작하는 객체의 내부 구현을 몰라야 한다}
    모두 객체라면 어떨까?
    \\~\\
    구조 감추기:
    \lstinputlisting[language=Java]{demeter/4.java}
    추상화 수준을 뒤섞어 놓음:
    \lstinputlisting[language=Java]{demeter/5.java}
    Clean Code:
    \lstinputlisting[language=Java]{demeter/6.java}
\end{frame}

\begin{frame}
    \frametitle{객체와 자료구조: 자료 전달 객체}
    \begin{itemize}
        \item 공개 변수만 있고 함수가 없는 클래스
        \item `Data Transfer Object' 로 불리기도 함
        \item 데이터베이스 통신이나 소켓 메시지를 파싱할 때 사용됨
    \end{itemize}
\end{frame}

\begin{frame}
    \frametitle{객체와 자료 구조: 결론}
    \begin{itemize}
        \item 객체는 행위를 드러내고, 데이터는 숨긴다
        \item 자료 구조는 데이터를 드러내고, 별다른 기능이 없다
        \item 객체에는 새로운 데이터 타입을 추가하기가 쉽다
        \item 자료 구조에는 새로운 기능을 추가하기가 쉽다
        \item 문제 해결을 위한 최적의 방식을 편견 없이 고르는 것이 중요
    \end{itemize}
\end{frame}

\end{document}
